\documentclass[aps,%
14pt,%
final,%
oneside,
onecolumn,%
musixtex, %
superscriptaddress,%
centertags]{extarticle} %% 
\usepackage[english,russian]{babel}
\usepackage[utf8]{inputenc}
%всякие настройки по желанию%
\usepackage[colorlinks=true,linkcolor=blue,unicode=true]{hyperref}
\usepackage{euscript}
\usepackage{supertabular}
\usepackage[pdftex]{graphicx}
\usepackage{amsthm,amssymb, amsmath}
\usepackage{textcomp}
\usepackage[bottom=20mm, top=20mm, left=30mm, right=15mm]{geometry}

\begin{document}

\begin{titlepage} 
\begin{center}
% Upper part of the page
{\large САНКТ-ПЕТЕРБУРГСКИЙ \\ ГОСУДАРСТВЕННЫЙ УНИВЕРСИТЕТ} \\[1.0cm]
{\large Математическое обеспечение и администрирование информационных систем} \\[0.2cm]
{\large Системное программирование} \\[3.5cm]
 
% Title
\textbf{\Large Назаренко Владимир Владимирович} \\[1cm]
\textbf{\LARGE Выделение объектов на видеопоследовательности}\\[1.0cm]
{\Large Выпускная квалификационная работа} \\[3.5cm]

%supervisor
\begin{flushright} \large
\emph{Научный руководитель:} \\
ст. преп. \textsc{Смирнов М. Н.}
\end{flushright}
 \begin{flushright} \large
\emph{Рецензент:} \\
\textsc{Пенкрат Н. А.} \\
менеджер проектов, ООО ``Ланит-Терком''
\end{flushright}
\vfill 

% Bottom of the page
{\large {Санкт-Петербург}} \par
{\large {2018 г.}}
\end{center} 
\end{titlepage}

\begin{titlepage} 
\begin{center}
% Upper part of the page
{\large SAINT PETERSBURG STATE UNIVERSITY} \\[1.0cm]
{\large Software and Administration of Information Systems} \\[0.2cm]
{\large Software Engineering} \\[3.5cm]
 
% Title
\textbf{\Large Vladimir Nazarenko} \\[1cm]
\textbf{\LARGE Object detection in a video sequence}\\[1.0cm]
{\Large Master thesis} \\[3.5cm]

%supervisor
\begin{flushright} \large
\emph{Scientific advisor:} \\
sr. Lecturer \textsc{Mikhail Smirnov}
\end{flushright}
 \begin{flushright} \large
\emph{Reviewer:} \\
\textsc{Nickolay Penkrat} \\
Project Manager, Lanit-Tercom LLC
\end{flushright}
\vfill 

% Bottom of the page
{\large {St Petersburg}} \par
{\large {2018}}
\end{center} 
\end{titlepage}


% Table of contents
\tableofcontents

\newpage

\section*{Введение}

% Анализ изображений

% Применение к ADAS и важность ADAS

% Почему не лидар

В настоящее время широкое распространение получили системы помощи водителю (ADAS). Такие системы, например, предупреждают водителя об ограничении скорости на участке дороге (с помощью детектирования соответствующих дорожных знаков), о пересечении маркеров дорожной разметки, об опасности столкновения с различными объектами.

В данной работе мы сфокусировались на разработке \textit{сенсора безопасного движения}, решающего две из перечисленных выще задач: предупреждение о столновениях и предупреждение о пересечении маркеров дорожной разметки. 

Для решения обозначенных выше задач автомобиль оснащается различными сенсорами. Наиболее распространёнными сенсорами для решения перечисленных выше задача являются лидары, радары и оптические системы видимого спектра. Область применения каждого из типов сенсоров ограничена. Так, радары обладают низкой точностью определения формы и расстояния до объекта. Лидары обладают низкой точностью в плохих погодных условиях. Кроме того, высокая стоимость лидаров ограничивает их применение в массовом сегменте автомобильной промышленности. Использование оптических систем требует использования ресурсоёмких алгоритмов для определения расстояния до объектов. В связи с высокой доступностью и возможностью получения высокой точности измерений, в данной работе в качестве сенсора мы выбрали оптическую систему, состоящую из двух откалиброванных камер видимого диапазона.

Существует, как минимум, два класса алгоритмов для решения задач помощи водителю, использующих оптические сенсоры: нейросетевые алгоритмы и алгоритмы на основе методов классического компьютерного зрения. В данной работе решено было использовать алгоритмы на основе классического компьютерного зрения. Связано это со следующими проблемами нейросетевых алгоритмов.
\begin{itemize}
\item Сложность модицикации нейросетевых алгоритмов.
\item Неуниверсальность нейросетевых алгоритмов.
\item Сложность интерпретации нейросетевых алгоритмов.
\end{itemize}

Также существует два подхода к извлечению информации о расстоянии до объектов из изображений: подход на основе рассчёта карты глубины и подход на основе вычисления оптического потока. Подход на основе оптического потока привлекателен тем, что для него требуется только одна камера, что делает систему более простой и надёжной. Однако подход на основе рассчёта карты глубины обеспечивает большую точность. В связи с этим в данной работе мы придерживаемся подхода на основе рассчёта карты глубины.

 Строго говоря, под предупреждением водителя о столкновении мы понимаем детектирование на изображении \textit{\textbf{препятствий}} -- любых объектов, которые делают невозможным или опасным проезд через занимаемую ими область пространства. Типовыми примерами препятствий являются люди, автомобили, столбы, здания. Также мы считаем препятствиями особенности рельефа (холмы) и различные мелкие объекты, такие как бордюры. Слова "объект" и "препятствие" для нас являются синонимами. 

Под \textit{\textbf{детектированием препятствий}} мы понимаем выделение препятствий на изображении одним из следующих способов.
\begin{itemize}
    \item В виде описывающего прямоугольника.
    \item В виде пикселей, принадлежащих объекту.
    \item В виде области на изображении, содержащей все объекты.
    \item В виде области на изображении, движение в которой безопасно (так называемая \textit{\textbf{безопасная область}}).
\end{itemize}
Также отметим, что термины ``выделение объектов'', ``детектирование объектов'' и ``сегментация изображения на объекты'' мы считаем эквивалентными.

Под \textit{\textbf{детектированием дорожной разметки}} мы понимаем задачу выделения следующих маркеров на дорожном полотне.
\begin{itemize}
    \item Сплошная линия.
    \item Двойная сплошная линия.
    \item Прерывистая линия.
\end{itemize}

\section{Постановка задачи}

Целью данной работы является разработка и реализация, на основе подходов классического стереозрения и классической обработки изображений, алгоритмов для "сенсора безопасного движения".
Для достижения этой цели в рамках работы были сформулированы следующие задачи.
\begin{itemize}
    \item Разработать и реализовать алгоритм поиска препятствий движению автомобиля на видеопоследовательности.
    \item Разработать и реализовать алгоритм поиска маркеров дорожной разметки на изображении.
    \item Провести апробацию разработанных алгоритмов.
\end{itemize}

\section{Обзор}
\section {Поиск препятствий движению автомобиля }

\section{Детектирование маркеров дорожной разметки}

\section{Апробация}

\subsection{Данные}

\subsection{Оценка качества работы алгоритмов}

\section{Заключение}

В рамках данной работы были достигнуты следующие результаты.

\begin{itemize}
\item На основе подхода Stixel World и классического стереозрения разработан и релизован на языке C++ алгоритм решения задачи поиска препятствий движению автомобиля на видеопоследовательности, полученной со стерео-камеры, закреплённой на лобовом стекле автомобиля.

\item На основе методов классической обработки изображений разработан и реализован на языке C++ алгоритм поиска маркеров дорожной разметки на изображении, полученном с камеры, закреплённой на лобовом стекле автомобиля.

\item Выполнена апробация разработанных алгоритмов на наборах данных KITTI и tuSimple, а также на собственных данных.

\end{itemize}




\newpage

\bibliographystyle{ugost2008ls}
\bibliography{diploma.bib}
\end{document}


